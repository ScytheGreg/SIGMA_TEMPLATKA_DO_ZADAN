% ============================================
%  main.tex — przykład użycia klasy sinol.cls
% ============================================

\documentclass{sinol}

% --------------------------------------------
% Dane nagłówka
% --------------------------------------------
\ustawAutor{Jan Kowalski}
\ustawPrzedmiot{Matematyka dyskretna}
\ustawSeria{DOM\_1}
\ustawNrZadania{1, 2}

% --------------------------------------------
% Konfiguracja znaków ASCII
% --------------------------------------------
\ustawZnakNaglowka{=}
\ustawZnakRamki{*}
\ustawZnakRozwiazania{-}

% ============================================
% TREŚĆ DOKUMENTU
% ============================================
\begin{document}
\begin{multicols}{2} % dwie kolumny A5 obok siebie

% ==== Zadanie nr 1 ====
\begin{zadanie}
1. Treść zadania.
\blindtext
\end{zadanie}

\begin{rozwiazanie}
\blindtext[2]
\end{rozwiazanie}

% ==== Zadanie nr 2 ====
\begin{zadanie}
2. Treść zadania.
\end{zadanie}

\begin{rozwiazanie}
\blindtext[1]
\end{rozwiazanie}

\end{multicols}
\end{document}
